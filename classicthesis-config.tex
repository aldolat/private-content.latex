\usepackage[utf8]{inputenc}
\usepackage[T1]{fontenc}

\usepackage[
	drafting=false,    % print version information on the bottom of the pages
	tocaligned=false, % the left column of the toc will be aligned (no indentation)
	dottedtoc=true,  % page numbers in ToC flushed right
	eulerchapternumbers=true, % use AMS Euler for chapter font (otherwise Palatino)
	linedheaders=false,       % chaper headers will have line above and beneath
	floatperchapter=true,     % numbering per chapter for all floats (i.e., Figure 1.1)
	eulermath=false,  % use awesome Euler fonts for mathematical formulae (only with pdfLaTeX)
	beramono=true,    % toggle a nice monospaced font (w/ bold)
	palatino=true,    % deactivate standard font for loading another one, see the last section at the end of this file for suggestions
	style=arsclassica % classicthesis, arsclassica
]{classicthesis}

\newcommand{\myTitle}{Private Content\xspace}
\newcommand{\mySubtitle}{User manual of the WordPress plugin\xspace}
\newcommand{\myName}{Aldo Latino\xspace}
\newcommand{\myLocation}{Italy\xspace}
\newcommand{\myTime}{February 2020\xspace}

\usepackage[american]{babel}

\usepackage{csquotes}

\usepackage{graphicx} %
\usepackage{caption}
\usepackage{subfig}
\graphicspath{{images/}}
%\captionsetup{labelformat=empty} % Rimuove la dicitura "Figura 1" dalle didascalie.
%\captionsetup[subfloat]{labelformat=empty} % Rimuove la numerazione alfabetica delle subfigure.
%\captionsetup{font=scriptsize}


\usepackage{scrhack} % fix warnings when using KOMA with listings package
\usepackage{xspace} % to get the spacing after macros right

\definecolor{code}{RGB}{233, 30, 99}

\usepackage{tabularx} % better tables
\setlength{\extrarowheight}{3pt} % increase table row height

\usepackage{textcomp}
\usepackage{listings}
%\lstset{emph={trueIndex,root},emphstyle=\color{BlueViolet}}%\underbar} % for special keywords
\lstset{
  language=[LaTeX]Tex,
  morekeywords={PassOptionsToPackage,selectlanguage},
  basicstyle=\small\ttfamily\color{code},
  keywordstyle=\color{RoyalBlue},%\bfseries,
  %identifierstyle=\color{NavyBlue},
  commentstyle=\color{Green}\ttfamily,
  stringstyle=\rmfamily,
  numbers=none,%left,%
  numberstyle=\scriptsize,%\tiny
  stepnumber=5,
  numbersep=8pt,
  showstringspaces=false,
  breaklines=true,
  %frameround=ftff,
  %frame=single,
  belowcaptionskip=.75\baselineskip,
  upquote=true
  %frame=L
}

\usepackage{classicthesis}

\hypersetup{%
  %draft, % hyperref's draft mode, for printing see below
  colorlinks=true, linktocpage=true, pdfstartpage=3, pdfstartview=FitV,%
  % uncomment the following line if you want to have black links (e.g., for printing)
  %colorlinks=false, linktocpage=false, pdfstartpage=3, pdfstartview=FitV, pdfborder={0 0 0},%
  breaklinks=true, pageanchor=true,%
  pdfpagemode=UseNone, %
  % pdfpagemode=UseOutlines,%
  plainpages=false, bookmarksnumbered, bookmarksopen=true, bookmarksopenlevel=1,%
  hypertexnames=true, pdfhighlight=/O,%nesting=true,%frenchlinks,%
  urlcolor=CTurl, linkcolor=CTlink, citecolor=CTcitation, %pagecolor=RoyalBlue,%
  %urlcolor=Black, linkcolor=Black, citecolor=Black, %pagecolor=Black,%
  pdftitle={\myTitle},%
  %pdfauthor={\textcopyright\ \myName, \myUni, \myFaculty},%
  pdfauthor={\textcopyright\ \myName},%
  pdfsubject={},%
  pdfkeywords={},%
  pdfcreator={pdfLaTeX},%
  pdfproducer={LaTeX with hyperref and ClassicThesis (ArsClassica)}%
}
