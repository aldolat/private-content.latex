\begingroup
	\footnotesize

	\noindent This file was written in \LaTeX~with \textit{ClassicThesis} and
	the \textit{ArsClassica} style. It contains the entire text of the Wiki with
	some adaptations. The wiki is available at:

	\begin{center}
		\url{https://github.com/aldolat/private-content/wiki}
	\end{center}

	\section*{Plugin development and download}

	\noindent The plugin is developed using:

	\begin{itemize}
		\item GNU/Linux operating system;
		\item \href{https://code.visualstudio.com}{Visual Studio Code} as development application;
		\item \href{https://git-scm.com}{Git}, as version control system;
		\item \href{https://github.com/aldolat/private-content}{GitHub} as development repository;
		\item \href{https://gnupg.org}{GnuPG} as signing commits application;
		\item Apache, MySQL and PHP as development platform;
	\end{itemize}

	and can be downloaded from the WordPress official repository:
	\begin{center}
	\url{https://wordpress.org/plugins/private-content}
	\end{center}

	\section*{Author contacts and forums}

	\noindent The author can be contacted via email at
	\href{mailto:aldolat@gmail.com}{aldolat@gmail.com}. For support, please use the
	official forums:
	\begin{center}
	\url{https://wordpress.org/support/plugin/private-content}
	\end{center}

	\section*{License}

	\noindent The plugin is released under the terms of the \textsc{gpl} License
	v3.0 and later:

	\begin{quote}
		\begin{center}
			Copyright \copyright~2009, 2020  Aldo Latino

			(email : \url{mailto:aldolat@gmail.com})
		\end{center}
		This program is free software: you can redistribute it and/or modify it
		under the terms of the \textsc{gnu} General Public License as published by
		the Free Software Foundation, either version 3 of the License, or (at your
		option) any later version.

		This program is distributed in the hope that it will be useful, but
		\textsc{without any warranty}; without even the implied warranty of
		\textsc{merchantability} or \textsc{fitness for a particular purpose}.  See
		the \textsc{gnu} General Public License for more details.

		You should have received a copy of the \textsc{gnu} General Public License
		along with this program. If not, see \url{http://www.gnu.org/licenses/}.
	\end{quote}

	\noindent This document is released under the same terms of the plugin's
	license.
\endgroup