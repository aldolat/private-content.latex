\begingroup
	\footnotesize

	\noindent This file was written in \LaTeX~with \textit{ClassicThesis} and
	the \textit{ArsClassica} style. It contains the entire text of the Wiki with
	some adaptations. The wiki is available at:

	\begin{center}
		\url{https://github.com/aldolat/private-content/wiki}
	\end{center}

	\section*{Plugin development and download}

	\noindent The plugin is developed using:

	\begin{itemize}
		\item GNU/Linux operating system;
		\item \href{https://code.visualstudio.com}{Visual Studio Code} as development application;
		\item \href{https://git-scm.com}{Git}, as version control system;
		\item \href{https://github.com/aldolat/private-content}{GitHub} as development repository;
		\item \href{https://gnupg.org}{GnuPG} as signing commits application;
		\item Apache, MySQL and PHP as development platform;
	\end{itemize}

	and can be downloaded from the WordPress official repository:
	\begin{center}
	\url{https://wordpress.org/plugins/private-content}
	\end{center}

	\section*{Author contacts and forums}

	\noindent The author can be contacted via email at
	\href{mailto:aldolat@gmail.com}{aldolat@gmail.com}. For support, please use the
	official forums:
	\begin{center}
	\url{https://wordpress.org/support/plugin/private-content}
	\end{center}

	\section*{Acknowledgments}

	Many thanks to all the user that contributed to the development of this
	plugin, in particular:

	\begin{itemize}
		\item
		\href{http://www.wprecipes.com/add-private-notes-to-your-wordpress-blog-posts}{Jean
		Baptiste Jung} (the link is currently dead) for the idea behind this
		plugin;
		\item \href{http://digwp.com/2010/05/private-content-posts-shortcode}{Jeff Starr}
		for the initial code;
		\item all the wonderful users that gave me ideas and tips for improving
		the plugin.
	\end{itemize}

	The photo \ref{fig:dolphin} is by \href{https://unsplash.com/@emerald_}{Noah
	Boyer} on \href{https://unsplash.com}{Unsplash}.

	\section*{License}

	Copyright \copyright~2009, 2020  Aldo Latino

	\noindent The plugin is released under the terms of the \textsc{gpl} License
	v3.0 and later:

	\begin{center}
		\url{http://www.gnu.org/licenses/gpl-3.0.html}
	\end{center}

	\noindent This document is released under the same terms of the plugin's
	license.
\endgroup