\begingroup
	\pdfbookmark[1]{\contentsname}{tableofcontents}
	\tableofcontents

	\newpage

	\let\clearpage\relax
    %\let\cleardoublepage\relax

	\pdfbookmark[1]{\listfigurename}{lof}
    \listoffigures

    \vspace{8ex}

    \pdfbookmark[1]{\listtablename}{lot}
    \listoftables

    \vspace{8ex}

    \pdfbookmark[1]{Acronyms}{acronyms}
    \markboth{\spacedlowsmallcaps{Acronyms}}{\spacedlowsmallcaps{Acronyms}}
    \chapter*{Acronyms}
    \begin{acronym}[HTML]
        \acro{HTML}{HyperText Markup Language}

        {\small Hypertext Markup Language (\textsc{html}) is the standard markup
        language for documents designed to be displayed in a web browser. It can
        be assisted by technologies such as Cascading Style Sheets
        (\textsc{css}) and scripting languages such as JavaScript (from
        \textit{Wikipedia}).}

        \acro{CSS}{Cascading Style Sheets}

        {\small Cascading Style Sheets (\textsc{css}) is a style sheet language
        used for describing the presentation of a document written in a markup
        language like \textsc{html}. \textsc{css} is a cornerstone technology of
        the World Wide Web, alongside \textsc{html} and JavaScript (from
        \textit{Wikipedia}).}

        \acro{PHP}{PHP: Hypertext Preprocessor}

        {\small \textsc{php} is a popular general-purpose scripting language
        that is especially suited to web development. It was originally created
        by Rasmus Lerdorf in 1994; the \textsc{php} reference implementation is
        now produced by The \textsc{php} Group. \textsc{php} originally stood
        for Personal Home Page, but it now stands for the recursive initialism
        \textsc{php}: Hypertext Preprocessor (from \textit{Wikipedia}).}
    \end{acronym}
\endgroup
